\documentclass[conference]{IEEEtran}
\usepackage{cite}

\title{UV-VIS Spectrophometer berbasis single beam}
\author{\IEEEauthorblockN{Muhammad Ogin Hasanuddin}
\IEEEauthorblockA{\textit{School of Electrical Engineering and Informatics}\\
\textit{Institut Teknologi Bandung}\\
Bandung, Indonesia\\
Email:}}


% ganti abstract menjadi absktrak
\renewcommand{\abstractname}{Abstrak}

% ganti index term menjadi Kata Kunci
\renewcommand{\IEEEkeywordsname}{Kata Kunci}


\begin{document}

% ini komentar
\maketitle

\begin{abstract}
    Instrumen yang telah dikembangkan kelompok Tugas Akhir Teknik Elektro sebelumnya adalah sebuah spektrofotometer portabel yang dapat bekerja pada spektrum gelombang ultraviolet dan cahaya tampak, atau umumnya disebut spektrofotometer UV-Vis. Spektrofotometer yang dibuat hanya dapat melakukan perhitungan intensitas cahaya secara relatif terhadap nilai intensitas tertinggi yang terukur, tidak dilakukan kalibrasi skala panjang gelombang dan tuning fungsi konversi citra. Hal ini menunjukkan bahwa pengukuran yang dilakukan masih bersifat kualitatif. Spektrofotometer yang telah dikembangkan kelompok Tugas Akhir Teknik Elektro belum memiliki kapabilitas untuk 
\end{abstract}

\begin{IEEEkeywords}
    UV-VIS Spectrophotometry
\end{IEEEkeywords}

% Pendahuluan
\section{Pendahuluan}
ini dalah pendahuluan
    
% Pendahuluan
\section{Pekerjaan Terkait}
ini dalah pendahulua~c\cite{gutierrez2012dispersion}

% Pendahuluan
\section{Desain Sistem}
ini dalah pendahuluan


% Pendahuluan
\section{Hasil dan Pembasahan}
ini dalah pendahuluan

% Pendahuluan
\section{Kesimpulan}
ini dalah pendahuluan

% Referensi
\bibliographystyle{IEEEtran}
\bibliography{referensi.bib}

\end{document}